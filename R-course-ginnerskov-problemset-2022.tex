% Options for packages loaded elsewhere
\PassOptionsToPackage{unicode}{hyperref}
\PassOptionsToPackage{hyphens}{url}
%
\documentclass[
]{article}
\usepackage{amsmath,amssymb}
\usepackage{lmodern}
\usepackage{iftex}
\ifPDFTeX
  \usepackage[T1]{fontenc}
  \usepackage[utf8]{inputenc}
  \usepackage{textcomp} % provide euro and other symbols
\else % if luatex or xetex
  \usepackage{unicode-math}
  \defaultfontfeatures{Scale=MatchLowercase}
  \defaultfontfeatures[\rmfamily]{Ligatures=TeX,Scale=1}
\fi
% Use upquote if available, for straight quotes in verbatim environments
\IfFileExists{upquote.sty}{\usepackage{upquote}}{}
\IfFileExists{microtype.sty}{% use microtype if available
  \usepackage[]{microtype}
  \UseMicrotypeSet[protrusion]{basicmath} % disable protrusion for tt fonts
}{}
\makeatletter
\@ifundefined{KOMAClassName}{% if non-KOMA class
  \IfFileExists{parskip.sty}{%
    \usepackage{parskip}
  }{% else
    \setlength{\parindent}{0pt}
    \setlength{\parskip}{6pt plus 2pt minus 1pt}}
}{% if KOMA class
  \KOMAoptions{parskip=half}}
\makeatother
\usepackage{xcolor}
\usepackage[margin=1in]{geometry}
\usepackage{graphicx}
\makeatletter
\def\maxwidth{\ifdim\Gin@nat@width>\linewidth\linewidth\else\Gin@nat@width\fi}
\def\maxheight{\ifdim\Gin@nat@height>\textheight\textheight\else\Gin@nat@height\fi}
\makeatother
% Scale images if necessary, so that they will not overflow the page
% margins by default, and it is still possible to overwrite the defaults
% using explicit options in \includegraphics[width, height, ...]{}
\setkeys{Gin}{width=\maxwidth,height=\maxheight,keepaspectratio}
% Set default figure placement to htbp
\makeatletter
\def\fps@figure{htbp}
\makeatother
\setlength{\emergencystretch}{3em} % prevent overfull lines
\providecommand{\tightlist}{%
  \setlength{\itemsep}{0pt}\setlength{\parskip}{0pt}}
\setcounter{secnumdepth}{-\maxdimen} % remove section numbering
\ifLuaTeX
  \usepackage{selnolig}  % disable illegal ligatures
\fi
\IfFileExists{bookmark.sty}{\usepackage{bookmark}}{\usepackage{hyperref}}
\IfFileExists{xurl.sty}{\usepackage{xurl}}{} % add URL line breaks if available
\urlstyle{same} % disable monospaced font for URLs
\hypersetup{
  pdftitle={Problem set 6 - Text Analysis},
  pdfauthor={Josef Ginnerskov},
  hidelinks,
  pdfcreator={LaTeX via pandoc}}

\title{Problem set 6 - Text Analysis}
\author{Josef Ginnerskov}
\date{2022-08-12}

\begin{document}
\maketitle

Depending on your level and your interests, you can either start with
completing the A tasks or go straight to the B tasks.

A. Preprocessing and basic statistics with tm 1. Load the tm package 2.
Generate a corpus object from a set of character vectors directly in
Rstudio by copying four short texts online like quotes, the first lines
of Wikipedia entries or tweets. 3. Preprocess the corpus in any way you
find fitting for you documents with the help of the tm\_map command but
see to remove stop words related to the language in which your quotes
are written. 4. Construct a document-term matrix from your corpus object
and take out the 10 most frequently occurring terms for each document.
5. Look into what words correlate with the 10th most frequently used
word in each document. 6. Generate a tf-idf object (i.e.~term
frequency--inverse document frequency) to inspect what words are most
distinct for each document.

B. Text analysis with tidy text 1. Load the tidy text package and the
gutenbergR package. Download the four books by the author Woolf,
Virginia by using the gutenberg\_download() command. 2. Convert your set
of books into the tidy text format and remove stop words with the list
provided in tidytext package. 3. Look up the 10 most frequently
occurring terms for each book. 4. Construct a tf-idf object (i.e.~term
frequency--inverse document frequency) and compare look up the five
words with the highest score for each document. Compare how and think
about why they differ from the 5 most frequently used words (task B.3).
5. Generate a word cloud on the corpus level (make sure to only include
the most frequently occurring terms to not overload your system and make
the graph interpretable). Note if any words not present in the results
given from task B.3 are present. 6. Run a LDA topic model with 6 topics
and inspect the prevalence of topics in the documents (the gamma).
Looking into the words constituting each topic (the beta) relates to
words in the tasks above.

\end{document}
